\documentclass[12pt]{beamer}
\usetheme{CambridgeUS}
\usepackage[utf8]{inputenc}
\usepackage{amsmath}
\usepackage{amsfonts}
\usepackage{booktabs}
\usepackage{amssymb}
\usepackage{bm}
\usepackage{graphicx}
\usepackage{listings}
\usepackage{comment}


%% code information
\lstset{frame=tb,
  language=Python,
  aboveskip=3mm,
  belowskip=3mm,
  showstringspaces=false,
  columns=flexible,
  basicstyle={\small\ttfamily},
  numbers=none,
  classoffset=1,
  morekeywords={True,False}, keywordstyle=\color{munsell}, 
  classoffset=0, 
  keywordstyle=\color{blue},  
  commentstyle=\color{dkgreen},
  stringstyle=\color{PUJ3},
  numberstyle=\tiny\color{gray},
  breaklines=true,
  breakatwhitespace=true,
  tabsize=4,
}

\author{Diego}
\title{Bankruptcy prediction using Random forest and Adaboost}
%\setbeamercovered{transparent} 
%\setbeamertemplate{navigation symbols}{} 
%\logo{} 
\institute{BIT} 
%\date{} 
\subject{Education and Quantile Regression} 
\begin{document}

\begin{frame}
\titlepage
\end{frame}

%\begin{frame}
%\tableofcontents
%\end{frame}


\begin{frame}{Introduction}
Several studies in Colombia evidence that sex, ethnic, and age could be important variables to determine the educational attainment.
\end{frame}


\begin{frame}{Justification}
Hanushek shows that economic growth rely on in human capital aggregate levels, this last measure with standardized test (PISA).
\end{frame}

\begin{frame}{Methodology}{Data}
This work uses the official available public data in the ICFES web site 
\end{frame}


\begin{frame}{Methodology}{Model}
\begin{equation}
\begin{align*}
y = \beta_{0} + \beta_{1}  +... + u_{i}
\end{align*}
\end{equation}
\end{frame}


\begin{frame}{Performance assessment}
% Please add the following required packages to your document preamble:
%
\begin{table}[]
\centering
\resizebox{5cm}{!}{%
\begin{tabular}{lll}
\midrule
Metric &                            &    \\
MAE    & $\Vert \hat{y} - y \Vert$  & 10 \\
RSE    & $ \sum (\hat{y} - y )^{2}$ & 11
\bottomrule
\end{tabular}
}
\caption{Own elaboration}
\label{tab:my-table}
\end{table}
\end{frame}


\begin{comment}


\begin{frame}
\begin{table}
\input{RESULTADOS.tex}
\end{table}
\end{frame}

\begin{frame}{Graph}
\includegraphics[scale=0.4]{scatter.eps}
According to the last figure there is a positive relation between var2 and var3.
\end{frame}


\end{comment}

\begin{frame}{Add equation inside text}
Suppose that $\bm{X} \sim N(0,1)$ therefore we can model with 

\end{frame}

\begin{frame}[fragile]{Code python implementation}
\begin{lstlisting}
for x in range(1,10):
	print(x)
\end{lstlisting}
\end{frame}

\begin{frame}[fragile]
\begin{verbatim}
Init K centroids as K points:
	Update centroids with mean of each variable
\end{verbatim}
\end{frame}


\end{document}